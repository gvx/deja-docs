\section{Instructions}

Each instruction exists of four bytes: one byte for the opcode and three bytes
for the argument.

How the argument is interpreted depends on the opcode it belongs to. If
it is signed, it is represented in two's complement.

\begin{tabular}{llp{6.5cm}}
\bfseries{Argument type} & \bfseries{Signed?} & \bfseries{Explanation} \\ \hline
Index & No & A numerical reference to one of the literals (see section \ref{literals}), starting at 0. \\
Offset & Yes & An offset to another code location in the current module. In this case, 0 refers to the next instruction. \\
Number & Yes & An integer value, available to Déjà Vu as a number. \\
None & N/A & Some opcodes do not take arguments. These three bytes will be ignored. \\
\end{tabular}

\begin{tabular}{llll}
\bfseries{Opcode} & \bfseries{Function} & \bfseries{Argument type} \\ \hline
0x00 & PUSH\textunderscore{}LITERAL & Index \\
0x01 & PUSH\textunderscore{}INTEGER & Number \\
0x02 & PUSH\textunderscore{}WORD & Index \\
0x03 & SET & Index \\
0x04 & SET\textunderscore{}LOCAL & Index \\
0x05 & SET\textunderscore{}GLOBAL & Index \\
0x06 & GET & Index \\
0x07 & GET\textunderscore{}GLOBAL & Index \\
0x10 & JMP & Offset \\
0x11 & JMPZ & Offset \\
0x12 & RETURN & None \\
0x13 & RECURSE & None \\
0x14 & JMPEQ & Offset \\
0x15 & JMPNE & Offset \\
0x20 & LABDA & Offset \\
0x21 & ENTER\textunderscore{}SCOPE & None \\
0x22 & LEAVE\textunderscore{}SCOPE & None \\
0x30 & NEW\textunderscore{}LIST & None \\
0x31 & POP\textunderscore{}FROM & None \\
0x32 & PUSH\textunderscore{}TO & None \\
0x33 & PUSH\textunderscore{}THROUGH & None \\
0x40 & DROP & None \\
0x41 & DUP & None \\
0x42 & SWAP & None \\
0x43 & ROT & None \\
0x44 & OVER & None \\
0x50 & LINE\textunderscore{}NUMBER & Number \\
0x52 & SOURCE\textunderscore{}FILE & Index \\
0x60 & ENTER\textunderscore{}ERRHAND & Offset \\
0x61 & LEAVE\textunderscore{}ERRHAND & None \\
0x62 & RAISE & None \\
0x63 & RERAISE & None \\
0x70 & NEW\textunderscore{}DICT & None \\
0x71 & HAS\textunderscore{}DICT & None \\
0x72 & GET\textunderscore{}DICT & None \\
0x73 & SET\textunderscore{}DICT & None \\
\end{tabular}

\subsection{PUSH\textunderscore{}LITERAL}
This opcode pushes a literal value to the stack.

\subsection{PUSH\textunderscore{}INTEGER}
This opcode pushes a literal integer to the stack. It performs the same
basic function as PUSH\textunderscore{}LITERAL, except that is narrower
in scope (it only handles integers representable in 24 bits) and avoids
an entry in the literals table, which would be relatively expensive. A
conforming compiler can choose not to generate PUSH\textunderscore{}INTEGER
and instead add all literal integers to the literals table and generate
PUSH\textunderscore{}LITERAL instructions instead.

\subsection{PUSH\textunderscore{}WORD}
This opcode first performs the same functionality as GET and then, if
the value retrieved is a function, that function is called.
