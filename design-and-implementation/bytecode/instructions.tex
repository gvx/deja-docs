\section{Instructions}

Each instruction exists of four bytes: one byte for the opcode and three bytes
for the argument.

How the argument is interpreted depends on the opcode it belongs to. If
it is signed, it is represented in two's complement.

\begin{tabular}{|l|l|p{6.5cm}|}
\hline
\bfseries{Argument type} & \bfseries{Signed?} & \bfseries{Explanation} \\ \hline
Index & No & A numerical reference to one of the literals (see section \ref{literals}), starting at 0. \\ \hline
Offset & Yes & An offset to another code location in the current module. A 0 refers to the next instruction. \\ \hline
Number & Yes & An integer value, available to Déjà Vu as a number. \\ \hline
Nothing & N/A & Some opcodes do not take arguments. These three bytes will be ignored. \\ \hline
\end{tabular}

\emph{table of opcodes here}

