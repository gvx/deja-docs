\section{Instructions}

Each instruction exists of four bytes: one byte for the opcode and three bytes
for the argument.

How the argument is interpreted depends on the opcode it belongs to. If
it is signed, it is represented in two's complement.

\begin{tabular}{llp{6.5cm}}
\bfseries{Argument type} & \bfseries{Signed?} & \bfseries{Explanation} \\ \hline
Index & No & A numerical reference to one of the literals (see section \ref{literals}), starting at 0. \\
Offset & Yes & An offset to another code location in the current module. In this case, 0 refers to the next instruction. \\
Number & Yes & An integer value, available to Déjà Vu as a number. \\
None & N/A & Some opcodes do not take arguments. These three bytes will be ignored. \\
\end{tabular}

\begin{tabular}{lll}
\bfseries{Opcode} & \bfseries{Function} & \bfseries{Argument type} \\ \hline
0x00 & Push literal & Index \\
0x01 & Push integer & Number \\
0x02 & Push word & Index \\
0x03 & Set & Index \\
0x04 & Set local & Index \\
0x05 & Set global & Index \\
0x06 & Get & Index \\
0x07 & Get global & Index \\
0x10 & Jump & Offset \\
0x11 & Jump if zero & Offset \\
0x12 & Return & None \\
0x13 & Recurse & None \\
0x14 & Jump if equal & Offset \\
0x15 & Jump if not equal & Offset \\
0x20 & Create labda & Offset \\
0x21 & Enter scope & None \\
0x22 & Leave scope & None \\
0x30 & New list & None \\
0x31 & Pop from & None \\
0x32 & Push to & None \\
0x33 & Push through & None \\
0x40 & Drop & None \\
0x41 & Dup & None \\
0x42 & Swap & None \\
0x43 & Rot & None \\
0x44 & Over & None \\
0x50 & Line number & Number \\
0x52 & Source file & Index \\
0x60 & Enter error handler & Offset \\
0x61 & Leave error handler & None \\
0x62 & Raise & None \\
0x63 & Reraise & None \\
0x70 & New dict & None \\
0x71 & Has dict & None \\
0x72 & Get dict & None \\
0x73 & Set dict & None \\
\end{tabular}

