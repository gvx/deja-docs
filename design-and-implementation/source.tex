\chapter{Source language}

\newcommand\grammar[1]{\underline{#1}}

\section{Grammar}
Déjà Vu follows the off-side rule, like Python. Like Python, it uses colons to signal the start of an indented block. Unlike Python, it doesn't support line continuations.

On the next page is the complete grammar of the Déjà Vu language (because indentation is hard to capture in BNF, just imagine all indentation is replaced with INDENT and DEDENT tokens before parsing --- which the compiler actually does do). If anything is wrong or unclear, please contact me and I will hopefully be able to fix it in the next edition of this document.

The rest of this page will be used to show the various escape characters you can use in string literals:

\begin{tabular}{ l l }
\textbf{Sequence} & \textbf{Character inserted}\\
\hline
\textbackslash{}\textbackslash{} & Backslash\\
\textbackslash{}q & Double quote\\
\textbackslash{}n & Newline\\
\textbackslash{}r & Carriage return\\
\textbackslash{}t & Tab\\
\textbackslash{}\{$n$\} & Unicode code point $n$ (decimal notation)\\
\end{tabular}

\newpage
\begin{bnf*}
\bnfprod{program}{\bnfes \bnfor \bnfpn{optional contents}}\\
\bnfprod{optional contents}{\bnfpn{element} \bnfor \bnfpn{element} \bnfsp \bnfpn{optional contents}}\\
\bnfprod{element}{\bnfpn{statement} \bnfor \bnfpn{eol}}\\
\bnfprod{statement}{\bnfpn{simple line} \bnfor \bnfpn{simple statement} \bnfor \bnfpn{if suite} \bnfor \bnfpn{try-catch suite}}\\
\bnfprod{eol}{\bnftd{optional whitespace} \bnfsp \bnfpn{optional comment} \bnfsp \bnftd{newline}}\\
\bnfprod{simple line}{\bnfpn{words} \bnfsp \bnfpn{eol}}\\
\bnfprod{simple statement}{\bnfpn{while} \bnfor \bnfpn{for} \bnfor \bnfpn{function definition} \bnfor \bnfpn{repeat}}\\
\bnfprod{if suite}{\bnfts{if} \bnfsp \bnftd{whitespace} \bnfsp \bnfpn{words} \bnfsp \bnfpn{block} \bnfsp \bnfpn{elseifs} \bnfsp \bnfpn{else}}\\
\bnfprod{try-catch suite}{\bnfts{try} \bnfsp \bnfpn{block} \bnfsp \bnfpn{catches}}\\
\bnfprod{optional comment}{\bnfes \bnfor \bnfts{\#} \bnfsp \bnftd{anything except newline}}\\
\bnfprod{words}{\bnfpn{word} \bnfor \bnfpn{word} \bnfsp \bnftd{whitespace} \bnfsp \bnfpn{words}}\\
\bnfprod{while}{\bnfts{while} \bnfsp \bnftd{whitespace} \bnfsp \bnfpn{words} \bnfsp \bnfpn{block}}\\
\bnfprod{for}{\bnfts{for} \bnfsp \bnftd{whitespace} \bnfsp \bnfpn{name} \bnfsp \bnftd{whitespace} \bnfsp \bnfpn{words} \bnfsp \bnfpn{block}}\\
\bnfprod{function definition}{\bnfpn{function name} \bnfsp \bnfpn{optional arguments} \bnfsp \bnfpn{block}}\\
\bnfprod{repeat}{\bnfts{repeat} \bnfsp \bnftd{whitespace} \bnfsp \bnfpn{words} \bnfsp \bnfpn{block}}\\
\bnfprod{elseifs}{\bnfes \bnfor \bnfpn{elseif} \bnfsp \bnfpn{elseifs}}\\
\bnfprod{elseif}{\bnfts{elseif} \bnfsp \bnftd{whitespace} \bnfsp \bnfpn{words} \bnfsp \bnfpn{block}}\\
\bnfprod{else}{\bnfes \bnfor \bnfts{else} \bnfsp \bnfpn{block}}\\
\bnfprod{catches}{\bnfes \bnfor \bnfpn{catch} \bnfsp \bnfpn{catches}}\\
\bnfprod{catch}{\bnfts{catch} \bnfsp \bnftd{whitespace} \bnfsp \bnfpn{arguments} \bnfsp \bnfpn{block}}\\
\bnfprod{word}{\bnfpn{literal} \bnfor \bnfpn{optional get} \bnfsp \bnfpn{name} \bnfor \bnfpn{optional get} \bnfsp \bnfpn{method call}}\\
\bnfprod{block}{\bnftd{optional whitespace} \bnfsp \bnfts{:} \bnfsp \bnfpn{eol} \bnfsp \bnftd{indent} \bnfsp \bnfpn{contents} \bnfsp \bnftd{dedent} \bnfsp \bnfpn{eol}}\\
\bnfprod{contents}{\bnfpn{optional contents} \bnfsp \bnfpn{statement} \bnfsp \bnfpn{optional contents}}\\
\bnfprod{name}{\bnftd{anything except ", \#, ! or whitespace, or : or @ at the front}}\\
\bnfprod{funtion name}{\bnfts{labda} \bnfor \bnfts{func} \bnfsp \bnftd{whitespace} \bnfsp \bnfpn{name} \bnfor \bnfts{local} \bnfsp \bnftd{whitespace} \bnfsp \bnfpn{name} \bnfor \bnfpn{name}}\\
\bnfprod{optional arguments}{\bnfes \bnfor \bnftd{whitespace} \bnfsp \bnfpn{arguments}}\\
\bnfprod{arguments}{\bnfpn{name} \bnfor \bnfpn{name} \bnfsp \bnftd{whitespace} \bnfsp \bnfpn{arguments}}\\
\bnfprod{literal}{\bnftd{literal number} \bnfor \bnfts{:} \bnfsp \bnfpn{name} \bnfor \bnfpn{string} \bnfor \bnfpn{fraction}} \\
\bnfprod{optional get}{\bnfes \bnfor \bnfts{@}}\\
\bnfprod{method call}{\bnfpn{optional name} \bnfsp \bnfts{!} \bnfsp \bnfpn{message}}\\
\bnfprod{string}{\bnfts{"} \bnfsp \bnftd{anything except " or newline} \bnfsp \bnfts{"}}\\
\bnfprod{fraction}{\bnftd{integer} \bnfsp \bnfts{/} \bnfsp \bnftd{positive integer}}\\
\bnfprod{optional name}{\bnfes \bnfor \bnfpn{name}}\\
\bnfprod{message}{\bnfes \bnfor \bnfpn{name} \bnfor \bnfpn{name} \bnfsp \bnfts{!} \bnfsp \bnfpn{message}}\\
\end{bnf*}
\newpage

\section{Idents}
To understand \dv, one must first understand idents. The word
``ident'' is short for ``identifier''. Idents are symbolic values,
and their most common use is as variable names. Variables in \dv{}
consist of three objects: the name (an ident), the context (a scope),
and the value (any type of value).

If you see a word in \dv{} source code that looks like an identifier or
keyword in other languages, it is usually a ``proper word''. Basically, it is made into an
ident, then it looks up the value that is coupled to that ident, and
then calls the value as a function.

\section{Syntactic suger}
\dv{} has a lot of syntactic sugar to make common idioms easier to write
and read. A simple example for that is the get-syntax: \verb!@something!
is equivalent to \verb!get :something!
