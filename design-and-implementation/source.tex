\chapter{Source language}

\newcommand\grammar[1]{\underline{#1}}

\section{Grammar}
Déjà Vu follows the off-side rule, like Python. Like Python, it uses colons to signal the start of an indented block. Unlike Python, it doesn't support line continuations.

On the next page is the complete grammar of the Déjà Vu language (because indentation is hard to capture in BNF, just imagine all indentation is replaced with INDENT and DEDENT tokens before parsing --- which the compiler actually does do). If anything is wrong or unclear, please contact me and I will hopefully be able to fix it in the next edition of this document.

The rest of this page will be used to show the various escape characters you can use in string literals:

\bigskip

\begin{tabular}{ l l }
\textbf{Sequence} & \textbf{Character inserted}\\
\hline
\textbackslash{}\textbackslash{} & Backslash\\
\textbackslash{}q & Double quote\\
\textbackslash{}n & Newline\\
\textbackslash{}r & Carriage return\\
\textbackslash{}t & Tab\\
\textbackslash{}\{$n$\} & Unicode code point $n$ (decimal notation)\\
\end{tabular}

\newpage
\begin{bnf*}
\bnfprod{program}{\bnfpn{optional contents}}\\
\bnfprod{optional contents}{\bnfes \bnfor \bnfpn{element} \bnfsp \bnfpn{optional contents}}\\
\bnfprod{element}{\bnfpn{statement} \bnfor \bnfpn{eol}}\\
\bnfprod{statement}{\bnfpn{simple line} \bnfor \bnfpn{simple statement} \bnfor \bnfpn{if suite} \bnfor \bnfpn{try-catch suite}}\\
\bnfprod{eol}{\bnftd{optional white space} \bnfsp \bnfpn{optional comment} \bnfsp \bnftd{newline}}\\
\bnfprod{simple line}{\bnfpn{words} \bnfsp \bnfpn{eol}}\\
\bnfprod{simple statement}{\bnfpn{while} \bnfor \bnfpn{for} \bnfor \bnfpn{function definition} \bnfor \bnfpn{repeat}}\\
\bnfprod{if suite}{\bnfts{if} \bnfsp \bnfpn{optional words} \bnfsp \bnfpn{block} \bnfsp \bnfpn{elseifs} \bnfsp \bnfpn{else}}\\
\bnfprod{try-catch suite}{\bnfts{try} \bnfsp \bnfpn{block} \bnfsp \bnfpn{catches}}\\
\bnfprod{optional comment}{\bnfes \bnfor \bnfts{\#} \bnfsp \bnftd{anything except newline}}\\
\bnfprod{optional words}{\bnfes \bnfor \bnftd{white space} \bnfsp \bnfpn{words}}\\
\bnfprod{words}{\bnfpn{word} \bnfor \bnfpn{word} \bnfsp \bnftd{white space} \bnfsp \bnfpn{words}}\\
\bnfprod{while}{\bnfts{while} \bnfsp \bnfpn{optional words} \bnfsp \bnfpn{block}}\\
\bnfprod{for}{\bnfts{for} \bnfsp \bnftd{white space} \bnfsp \bnfpn{name} \bnfsp \bnfpn{optional words} \bnfsp \bnfpn{block}}\\
\bnfprod{function definition}{\bnfpn{function declaration} \bnfsp \bnfpn{optional arguments} \bnfsp \bnfpn{block}}\\
\bnfprod{repeat}{\bnfts{repeat} \bnfsp \bnfpn{optional words} \bnfsp \bnfpn{block}}\\
\bnfprod{elseifs}{\bnfes \bnfor \bnfpn{elseif} \bnfsp \bnfpn{elseifs}}\\
\bnfprod{elseif}{\bnfts{elseif} \bnfsp \bnfpn{optional words} \bnfsp \bnfpn{block}}\\
\bnfprod{else}{\bnfes \bnfor \bnfts{else} \bnfsp \bnfpn{block}}\\
\bnfprod{catches}{\bnfes \bnfor \bnfpn{catch} \bnfsp \bnfpn{catches}}\\
\bnfprod{catch}{\bnfts{catch} \bnfsp \bnftd{white space} \bnfsp \bnfpn{arguments} \bnfsp \bnfpn{block}}\\
\bnfprod{word}{\bnfpn{literal} \bnfor \bnfpn{optional get} \bnfsp \bnfpn{name} \bnfor \bnfpn{optional get} \bnfsp \bnfpn{method call}}\\
\bnfprod{block}{\bnftd{optional white space} \bnfsp \bnfts{:} \bnfsp \bnfpn{eol} \bnfsp \bnftd{indent} \bnfsp \bnfpn{contents} \bnfsp \bnftd{dedent}}\\
\bnfprod{contents}{\bnfpn{optional contents} \bnfsp \bnfpn{statement} \bnfsp \bnfpn{optional contents}}\\
\bnfprod{name}{\bnftd{anything except ", \#, ! or white space, or : or @ at the front}}\\
\bnfprod{function declaration}{\bnfts{labda} \bnfor \bnfts{local} \bnfsp \bnftd{white space} \bnfsp \bnfpn{name} \bnfor \bnfpn{function name}}\\
\bnfprod{function name}{\bnfts{func} \bnfsp \bnftd{white space} \bnfsp \bnfpn{name or method} \bnfor \bnfpn {name or method}}\\
\bnfprod{name or method}{\bnfpn{name} \bnfor \bnfpn{name} \bnfsp \bnfts{!} \bnfsp \bnfpn{name}}\\
\bnfprod{optional arguments}{\bnfes \bnfor \bnftd{white space} \bnfsp \bnfpn{arguments}}\\
\bnfprod{arguments}{\bnfpn{name} \bnfor \bnfpn{name} \bnfsp \bnftd{white space} \bnfsp \bnfpn{arguments}}\\
\bnfprod{literal}{\bnftd{literal number} \bnfor \bnfts{:} \bnfsp \bnfpn{name} \bnfor \bnfpn{string} \bnfor \bnfpn{fraction}} \\
\bnfprod{optional get}{\bnfes \bnfor \bnfts{@}}\\
\bnfprod{method call}{\bnfpn{optional name} \bnfsp \bnfts{!} \bnfsp \bnfpn{message}}\\
\bnfprod{string}{\bnfts{"} \bnfsp \bnftd{anything except " or newline} \bnfsp \bnfts{"}}\\
\bnfprod{fraction}{\bnftd{integer} \bnfsp \bnfts{/} \bnfsp \bnftd{positive integer}}\\
\bnfprod{optional name}{\bnfes \bnfor \bnfpn{name}}\\
\bnfprod{message}{\bnfes \bnfor \bnfpn{name} \bnfor \bnfpn{name} \bnfsp \bnfts{!} \bnfsp \bnfpn{message}}\\
\end{bnf*}
\newpage

\section{Semantics}
\subsection{Signatures}
\dv{}, being a stack-based language, has as a central feature a stack of
values. Values can be passed around, but can also be ``called'', in which case
they (among other things) manipulate the stack. The \defini{signature} of a
value is an indication of that stack manipulation. Most types have a signature
of \sig{0}{1}, that is: they take no values off the stack, and push a single
value on it (namely, themselves). The exceptions are called ``functions''.
Some examples from the standard library\footnote{See chapter \ref{chap:stdlib} starting from page \pageref{chap:stdlib} for an explanation of
each of those functions.}: \verb!drop! is \sig{1}{0}, \verb!dup!
is \sig{1}{2} and \verb!swap! is \sig{2}{2}. Some function have a variable
signature, that is: the number of arguments the function takes or the number of
values it returns can change per call. One such an example is \verb![! with
\sig{$n$}{1}, another is \verb!rep! with \sig{2}{$n$}.

Determining the signature of an arbitrary function is an undecidable\footnote{The
proof is left as an exercise for the reader} problem. This means that we
cannot, in general, know the signature without running it. And \emph{that}
means signatures are only used as hints for programmers.\footnote{I wrote a
variation on \dv{} once, where signatures were statically determined from
function bodies and enforced during execution. It was horrible to use.}

\subsection{Use--mention distinction}
blah blah two use--mention distinctions (get vs call and ident vs call)

\subsection{Statements and other line oriented subjects}
\subsubsection{Comments}

A single pound sign (\verb!#!) appearing anywhere except inside a string
literal starts a \defini{comment}. The following text will assume all the
comments have been stripped beforehand.

\subsubsection{Indentation}

Like in Python, in \dv{} \defini{indentation} is used to group code into
blocks. Lines that do not contain any code are ignored for indentation.
Lines that do contain code are called \defini{non-empty lines}.

The first non-empty line must have no leading white space. Only the first
non-empty line after a \defini{statement line} (lines ending in a colon)
can (and must) have more leading white space than the preceding non-empty
line. The leading white space of that statement line must be a prefix of
the leading white space of the line with increased indentation.

A non-empty line following a simple line (see below) can
have the same leading white space as preceding non-empty line, or the same as
a non-empty statement line that opened a block that (directly or indirectly)
contains the preceding non-empty line.

\subsubsection{Simple line}

Any non-empty line that is not a statement line is a \defini{simple line}.
A simple line is a sequence of \defini{words}, separated by white space.

Execution order is right to left. This is different from Forth and other
stack-based languages, but it is a better match for mathematical
notation. For example, $f(x)$ would be \verb!f x! in \dv{}.

Some statements have a part of their statement line where you can put arbitrary
words, which works almost the same as a simple line. The exception is that they,
unlike simple lines, can be empty.

\subsubsection{While}

A while-loop executes its body a number of times, much like the repeat-loop and
the for-loop. The part of its statement after the first word (\verb!while!) and
before the colon is a simple line that is executed before each iteration (a
\defini{conditional line}).

A single value is popped of the stack after running the conditional line, and
if it is truthy (see ...?) the body is run, and execution goes back to the
start. If that value was falsy, the loop is ended.

The conditional line is generally \sig{0}{1}, and less often \sig{$n$}{($n$+1)}
but that is not required. It can even be empty, as long as the code before and inside the loop makes sure there is at least one value on the stack before the loop
condition is checked.

\subsubsection{Repeat}

A repeat-loop executes its body a fixed number of times. The simple line
between its first word (\verb!repeat!) and the colon is executed for the loop
and is expected to push a number to the stack. That number is the number
of times the body is executed.

\subsubsection{For}
\subsubsection{If}

If-suites consist of an if-clause, followed by zero or more elseif-clauses, and then an optional else-clause. Every one of those has a conditional line, except for the else.

\subsubsection{Try--catch}

Each catch-clause lists the errors it handles.

Currently, there are the following types of error you can handle:

\begin{itemize}
	\item \verb!name-error! Tried to get or push a name which has no binding in the current context.
	\item \verb!value-error! An argument has an invalid value for the operation attempted.
	\item \verb!type-error! An argument has the wrong type.
	\item \verb!stack-empty! There are not enough values on the stack.
	\item \verb!illegal-file! A source file has a syntax error or a byte-code file is malformed.
	\item \verb!unicode-error! Attempted to make a string from a source that is not valid UTF-8.
	\item \verb!interrupt! An interrupt was sent to the virtual machine (often by pressing Ctrl+C in the terminal).
	\item \verb!error! A generic error type. Avoid if possible.
\end{itemize}

\subsubsection{Function definition}
\subsection{Words}
\section{Idents}
To understand \dv, one must first understand idents. The word
``ident'' is short for ``identifier''. Idents are symbolic values,
and their most common use is as variable names. Variables in \dv{}
consist of three objects: the name (an ident), the context (a scope),
and the value (any type of value).

If you see a word in \dv{} source code that looks like an identifier or
keyword in other languages, it is usually a ``proper word''. Basically, it is made into an
ident, then it looks up the value that is coupled to that ident, and
then calls the value as a function.

\section{Syntactic suger}
\dv{} has a lot of syntactic sugar to make common idioms easier to write
and read. A simple example for that is the get-syntax: \verb!@something!
is equivalent to \verb!get :something!

\section{Truthyness}

In \dv{}, all objects can be used in a boolean context. The only things that
count as falsy are: numbers equal to zero (this includes \verb!false!), empty
lists and dictionaries, and zero-length strings and blobs. In particular, idents
and functions are always truthy.

\section{Equality}

\dv{} has a single concept of \defini{equality}, that is used with both
\verb!=! and to distinguish dictionary keys from each other. A value is always
equal to itself (that is \verb!= dup! should never return \verb!false!). Two values of different types are never equal to each other.

For immutable types (strings, numbers, fractions and pairs), different values
that represent the same data are equal as well.

Idents are immutable as well, but since they are globally unique, there are never two separate instances of idents that would compare equal.
